\capitulo{2}{Objetivos del proyecto}

Para marcar los objetivos, tanto generales como los de la aplicación, analizamos las propuestas y conclusiones marcadas por Mario López Jiménez y Bruno Martím Gómez en sus respectivos proyectos.

\begin{itemize}
	\item  \textbf{UBUDiabetes: Aplicación de control de diabetes en dispositivos móviles}\cite{mario2016}.
	\begin{itemize}
		\item Añadir la posibilidad de exportar los datos almacenados en la base de datos SQLite de la aplicación en algún formato que pueda resultar útil para el usuario.
	\item Implementar la opción de borrar o modificar entradas de la base de datos.
	\item Adaptar la aplicación para \textit{tablets} aprovechando el diseño de la misma en \textit{fragments}.
	\item Añadir la opción de crear diferentes cuentas de usuario. Es una opción que en un teléfono no resulta especialmente útil, pero si podría serlo en caso de migrar la aplicación a  \texttt{tablets} que puedan ser utilizadas por más de un usuario.
	\item La biblioteca \textit{MPAndroidChart} está en continua evolución. Por lo tanto actualizar la versión de la misma y, en caso de que tuviera funcionalidades nuevas útiles, aprovecharlas en la aplicación.
	\item Ampliar la base de datos de alimentos y en caso de ser necesario si el numero es muy elevado, adaptar el modo en que 	estos se seleccionan por pantalla a las necesidades.
	\item Añadir más idiomas a la aplicación duplicando para ello los archivos \textit{xml} en los que se encuentran los \textit{Strings} de la misma.
	\end{itemize}
	
	\item \textbf{Evaluación de la eficacia, efectividad y usabilidad de la aplicación para móviles UBUdiabetes}\cite{bruno2017}.
	\begin{itemize}
		\item La posibilidad de mostrar los alimentos que se van introduciendo para el cálculo de hidratos de carbono en una comida.
		\item Ampliar la lista de alimentos disponibles. Y mostrar la relación de hidratos de carbono.
		\item Dar la posibilidad al usuario de elegir las unidades en las que quieren hacer el recuento (raciones, gramos o directamente los hidratos de carbono que los usuarios sacamos de las etiquetas de los productos que consumen).
		\item Incluir en la configuración del perfil del usuario bloques de tiempo para la configuración de distintos valores del Ratio.
		\item Añadir decimales a las unidades de insulina, ya que usuarios con mucha sensibilidad a la insulina, media unidad 	puede ser determinante.
		\item Tener en cuenta la insulina residual que hay en el organismo.
	\end{itemize}
\end{itemize}
Una vez analizadas las propuestas y/o conclusiones anteriores, procedemos a marcar los objetivos principales en este proyecto.
\section{ Objetivos Generales}
	\begin{itemize}
		\item Estudio y análisis del lenguaje y herramientas necesarios para el desarrollo de una nueva versión de la aplicación UBUDiabetes 1.1.
		\item Investigar herramientas para el desarrollo de prototipado.
		\item Estudiar las opciones que ofrece Android para la automatizacion de las distintas pruebas.
		
	\end{itemize}

\section{ Objetivos de la aplicación}
	\begin{itemize}
		\item Mejorar toda la interfaz gráfica de usuario.
		\item Añadir decimales a las unidades de insulina.
		\item Mostrar los alimentos que se van introduciendo para el cálculo de hidratos de carbono en una comida.
		\item Posibilidad de eliminar los alimentos introducidos para el cálculo de hidratos de carbono en una comida. 
		\item Mostrar registros diarios del usuario.		
		\item Ampliar la lista de alimentos disponibles.
		\item Mejorar la gráfica del historial de glucemias registradas.
		\item Incluir Pruebas Unitarias.
		\item Incluir pruebas Instrumentales-Espresso.
	\end{itemize}