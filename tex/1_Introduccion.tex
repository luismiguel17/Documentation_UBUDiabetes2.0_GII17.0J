\capitulo{1}{Introducción}
Actualmente podemos encontrar más de 1500 apps relacionadas con la diabetes en plataformas como Google Play, iTunes, Windows Market o BlackBery World. La mayoría se clasifican como aplicaciones de seguimiento de la enfermedad (33\%), ya que permiten el control de los registros de la insulina, glucemia, peso, condición física y conteo de carbohidratos. En un porcentaje menor (22\%), encontramos las apps educativas, que facilitan el recuento de carbohidratos mediante juegos o el cálculo de insulina a través de la glucemia. También existen apps nutricionales, centradas en el cálculo de carbohidratos en las comidas (8\%). Otras, en su mayoría pertenecientes a farmacéuticas, prestan información sobre sus productos (8\%). Y, por último, también se han desarrollado redes sociales y foros que ayudan a compartir experiencias entre las personas diabéticas (5\%).

UBUDiabetes 1.0 es una aplicación móvil dirigida a persona que padecen diabetes de tipo 1, que se desarrolló de la mano de Mario López Jiménez, autor del TFG ``UBUDiabetes: Aplicación de control de diabetes en dispositivos móviles'', marcado por las ideas surgidas de Lara Bartolomé Casado, autora del TFG ``Salud electrónica y Diabetes: Elementos para el diseño de una aplicación informática (App) para jóvenes diabéticos de tipo 1''.

Tras finalizar una primera versión de la aplicación, Raúl Marticorena Sánchez introdujo algunas modificaciones posteriores, las cuales llevaron a una nueva versión de la aplicación: UBUDiabetes 1.1

Desde el Grado de Enfermería, Bruno Martín Gómez, autor del TFG ``Evaluación de la eficacia, efectividad y usabilidad de la aplicación para móviles UBUDiabetes'', durante el curso 2016-2017 vio la necesidad de validar la eficacia y efectividad de la aplicación móvil UBUDiabetes 1.0 y posteriormente UBUDiabetes 1.1 abordando los siguientes objetivos\cite{bruno2017}:
\begin{itemize}
	\item Validar la eficacia \texttt{in vitro} de las funciones utilizadas en los algoritmos y comprobar los cálculos de hidratos de carbono y del bolo de insulina se realizan de forma correcta.
	\item Validar la efectividad in vivo de UBUdiabetes 1.1 comparando los datos obtenidos por los usuarios con la app y los que finalmente se han dosificado por su prescripción.
	\item Valorar el grado de satisfacción de los usuarios con la app y recoger sugerencias de mejora.
\end{itemize}
Tras una primera la validación de UBUDiabetes 1.1, Bruno Martín llegó a las siguientes conclusiones\cite{bruno2017}:
\begin{itemize}
	\item La eficiencia ha resultado satisfactoria en cuanto la mitad de los usuarios han obtenido resultados acertados respecto a la recomendación del bolo de insulina, y en la otra mitad, los cálculos no han sido tan precisos debido a que los usuarios no tenían la suficiente preparación para el recuento de alimentos. Este factor externo a la app implica que la versión actual de UBUDiabetes 1.1 debe ser recomendada a los usuarios adecuados, con un nivel apropiado de conteo de alimentos.
	\item Los usuarios han quedado satisfechos con la participación en el proyecto. Coinciden en que es sencilla, útil y de fácil manejo.
	\item UBUdiabetes 1.1 tiene que evolucionar, para alcanzar un mayor nivel de interacción con los usuarios a la hora del conteo de los alimentos y la visualización de sus registros diarios.
\end{itemize}

UBUDiabetes 2.0 se centra en la revisión y modificación de la versión 1.1, incluyendo nuevos requisitos surgidos por los compañeros de enfermería, así como en la inclusión de la automatización de las distintas pruebas: unitarias, integración, sistema, etc. 