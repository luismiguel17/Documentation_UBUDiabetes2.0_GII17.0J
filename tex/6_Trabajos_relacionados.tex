\capitulo{6}{Trabajos relacionados}
A continuación, se hará un breve resumen acerca de los trabajos relacionados con este proyecto. Cabe mencionar que se puede consultar más información acerca de este apartado en el trabajo de fin de grado realizado por Lara Bartolomé Casado~\cite{larab2015}.
Como bien menciona Lara, en la actualidad existen alrededor de unas 97.000 aplicaciones móviles de salud disponible para su descarga. Dentro de estas aplicaciones, la mayoría tratan sobre enfermedades endocrinas como la Diabetes.  La mayoría de estas ayudan a los usuarios a controlar la dosis de insulina o medicación, la glucemia capilar, o apoyan las tareas de autocuidado, como el ejercicio o la dieta, es decir, funcionas como aplicaciones de``gestión de la diabetes''. 
Algunas de las aplicaciones con mayor interés para los usuarios y que se recogen en el TFG de Lara, son los siguientes:
\begin{itemize}
	\item \textbf{Fundación para la diabetes}: aplicación para iOS y Android en la que se puede consultar recetas para diabéticos, realizar el test Frindisk y leer las ultimas noticias respecto a la diabetes.
	\item \textbf{Diario de la diabetes}: Aplicación para iOS y Android que permite registrar todos los datos importantes de la vida diaria con la diabetes.
	\item \textbf{Diguan}: Juego para iOS y Android con el que los niños se familiarizan con la enfermedad.
	\item \textbf{Diabetes}: La aplicación realiza un seguimiento de todos los aspectos del tratamiento de la enfermedad y proporciona informes detallados, gráficos y estadísticas para compartir a través del correo electrónico.
	\item \textbf{Diabetes hipoglicemia}: Guía para resolver una hipoglucemia de forma adecuada, hasta la resolución del problema.
	\item \textbf{Socialdiabetes}: Aplicación para iOS y Android. En ella se puede introducir las comidas, el nivel de glucosa o la cantidad de insulina. Tiene la opción de consultar las gráficas para seguir una evolución de forma visual.
\end{itemize}


